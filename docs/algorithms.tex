% v1.0 released 28th January 1994

\documentclass[a4paper]{article}
\usepackage[margin=1.0in, top = 1.2in, bottom=1.4in]{geometry}
%\usepackage{amsmath}
\usepackage[fleqn]{amsmath}
\usepackage{mathtools}
%\usepackage{natbib}
\usepackage{graphicx}
\usepackage{mathrsfs}
%\usepackage[usenames,dvipsnames]{color}
%\usepackage{aas_macros}
\usepackage{amssymb}
\usepackage{multirow}
%\usepackage{comment}
%\usepackage{subfigure}
\usepackage{hyperref}
%\usepackage{epstopdf, dcolumn}
\usepackage{color}
%\DeclareGraphicsExtensions{.eps}

\usepackage{fancyvrb}

\newcommand{\Rom}[1]{\uppercase\expandafter{\romannumeral #1}}
\newcommand{\rom}[1]{\lowercase\expandafter{\romannumeral #1}}
\newcommand{\Msun}{\mathrm{M_\odot}}
\newcommand{\hl}[1]{{\color{blue}#1}} 
%%%%%%%%%%%%%%%%%%%%%%%%%%%%%%%%%%%%%%%%%%%%%%%%

\begin{document}
\title{\Huge Pack Allocation Algorithms}

%\begin{document}

\date{\today}


\maketitle

\label{firstpage}
\section{Description of Problem}

Allocate the number of bakery products $m$ to a finite types of packs with sizes
(number of units) $V_i$ for $i=1, 2, ... n$. The allocation result can be described by the number of allocated packs $a_i$ for $i=1, 2, ... n$. An optimal solution is to minimize total number of required packs
$$B = \sum_{i=1}^n a_i,$$
subject to
\begin{itemize}
   \item $v_i > 0$ is a positive integer
   \item $a_i \geq 0$ is a non-negative integer
   \item $\sum_{i=0}^n a_i V_i = m$, where $m$ is the quantity of products to be allocated
\end{itemize}
The total price of the products can be calculated as
$$P = \sum_{i=1}^n p_i a_i,$$ where $p_i$ is the price of pack $i$.
\section{Algorithms}
This allocation problem is NP-hard. An exhaustive search is computationally expensive. I have therefore proposed a first-fit greedy approximation algorithm, which is verified with the exhaustive algorithm.

\subsection{Greedy Heuristic}
This greedy heuristic always try to allocate the items to the largest possible pack. If this is a reminder, try to allocate the reminder to the next smaller pack until the smallest. If it is still not allocable, remove one large pack and add the items $V_i$ to the reminder and try again above allocation process. The greedy algorithm is implemented in a recursive manner:
\begin{enumerate}
\item Sort the packs by their sizes, so that $V_0 > V_1 > .. V_n$.
\item Start from the largest pack $i=0$. Initialise $r=m$.
\item Divide $r$ by $V_i$, set quotient to $a_i$ and the reminder to$r$.
\item If $r = 0$, return the allocation result.
\item If $r > 0$, try next smaller pack $i = i + 1$, go to Step 3
\item If still not divisible and $a_i>0$ remove 1 current pack and add items to $r$ and repeat Step 3 - 6 until $a_i = 0$
\item If still $r>0$ at the end. The products are not allocable to these packs.
\end{enumerate}


\subsection{Exhaustive Search}
The exhaustive algorithm is also implemented in a recursive manner. For all $i=1, 2, .., n$, try $a_i = 0, 1, ...,\mathrm{reminder}[r/V_i]$. At the end of each step, subtract allocated items from the reminder $r$, this reduces the upper bound for next loop.

\section{Verification and Test} 
 
\subsection{User Cases}
The greedy heuristic is tested against the test cases from the specification. See Table 1 for the example user test cases.
\begin{table}
\label{tab:example-test}
 \caption{Example test cases}
 \centering
 \begin{tabular}{| c | c | c | c | c |}
 \hline
        product code & quantity ($m$) & pack sizes ($v_i$) & allocation ($a_i$) & test result\\
 \hline
 		VS5 & 10 & [5, 3] & [2, 0] & passed \\
 		MB11 & 14 & [8, 5, 2] & [1, 0, 3] & passed\\
 		CF & 13 & [9, 5, 3] & [0, 2, 1] & passed\\
  \hline
 \end{tabular}
\end{table}

\subsection{Verification}

The completeness the greedy heuristic is validated with the exhaustive algorithm. Two criteria are tested for $m$ in $1, 2, ... 100$.
\begin{itemize}
\item If the exhaustive finds at least a solution, the greedy heuristic also finds one. \item If the greedy heuristic doest not find a solution, no solution can be found by exhaustive search.
\end{itemize}

Specially, under configuration in Table {\ref{tab:example-test}}. I have also verified that the greedy heuristic is able to find the optimal solution.


\section{Performance}

\subsection{Accuracy}

The greedy method is a heuristic. It allocates the products to packs from large to small and return the allocation results immediately if this is one. For example, Two optimal allocation methods for MB11 under configuration [8, 5, 2] are [1, 0, 3] and [0, 2, 2]. The greedy method is able to find the first one.

Under more complex pack configuration, the greedy method may not give the best solution. for example, If $V$ = [6, 5, 2] and $m=10$. The heuristic gives [1, 0, 2]. The exhaustive search can find the optimal allocation [0, 2, 0]. However, it has been tested that under current pack configuration, the greedy heuristic is able to find one of the optimal allocation solutions.

\subsection{Time complexity}

\begin{figure}
\label{fig:comparison}
    \includegraphics[width=\textwidth]{./comparison.pdf}
    \caption{The speed performance of the greedy approximation and the exhaustive search}
\end{figure}

I have compared the time performance between the greedy heuristic and the exhaustive search. As shown in Figure 1,
The run time of the exhaustive algorithm increases almost exponentially with the the order quantity increases. If $m=1000$, it requires $\sim 36s$ to complete the run using my machine, which is inapplicable and the greedy approximation heuristic becomes a must. It is noticed that under the current pack configuration, the time complexity of the greedy heuristic is $\sim O(1)$.
\end{document}
